
%%%%%%%%% SUMMARY -- 1 page, third person
% e.g:  "The PI will prove" not "I will prove"

%Introduction / Summary (1 page).  The following information is typically included in proposal introductions for this 
%grant. The summary should lead to the proposal body but it should also be a self-contained document, 
%summarizing what is being proposed and why.   
 
% Provide appropriate background that sets the context for the problem and the objectives. 
%State the specific problem (may also be thought of as a need, central research question, or hypothesis) 
%your research will address. 
% State the objectives of the proposed research. 
%State any limits of the proposed research  (may not be needed). 
%Summarize the research methods you will use to achieve research objectives (if clearly covered elsewhere 
%in the introduction, a separate section may not be needed). 


%\required{Project Summary}
\section{Project Summary}
% This should be a brief statement of the problem you plan to address.
% It should look something like an abstract. 
In biomechanics, there are many interactions between the body and brain. Some
interactions can be quantified as electrical signals to motor nerves of muscles.
The result is a particular musclular reaction due to an electrical signal. The
input is an electrical nerve signal, and the output is a measured velocity at
a joint. The overall result to this sytem of inputs and output is actions like 
walking.

In the proposed project, the focus will be on how to relate these inputs to the 
outputs. Each muscular system adapts to what each input signal should mean, so
predicting the output for each system becomes a difficult challenge. The 
project will focus on optimizing solutions that will explain test data. These
solution could then be used for predicting outputs for inputs that have not
yet been measured. This would result in a better understanding of treatment 
of muscle problems, and better simulations for fixes of muscle irregularities.

\subsection{Background}
Walking requires a lot of complex tasks to make locomotion possible 
\cite{mcgowan}. The inputs to muscle motor nerves and the outputs of joint
motion are one way of measuring all these complex tasks. The data collected
as inputs and outputs fits nicely into Evolutionary Computation (EC). EC uses 
many measured inputs and expected outputs, and evolves a system that will 
produce those expected outputs on its own. 

The EC program will do this with some margin of error, of course, but the 
error of the solution should be small. The EC solution will then be useful 
when there are a set of inputs that the output is not already know, like in 
a simulation. The proposed project, two kinds of EC will be considered: 
Genetic Algorithms and Genetic Programming. Each has its strength and 
drawbacks, at least one approach will be used to optimize a model for 
simulation.

\subsection{Problem Statement}
For Dr. Craig McGowan's research, a framework already exists. It has been a 
collaboration between many researchers for many years. There are two problems
that the research faces, that the project will focus on: optimizing models 
for the current evaluation function, and creating a better evaluation 
function. A solution for the first problem will be attempted, and a solution
for the second will be attempted if time remains. 

McGowan's research has produced their own evaluation function for how 
well a simulated biomechanical model is walking. The function is very large and abstract,
although supported by research. It has many parameters itself that are educated
guesses. McGowan then produces many semi-random models, and uses Simulated 
Annealing optimization to find the best model per the evaluation function. 

\subsection{Objectives}
The project will use a Genetic Algorithm (GA) for the first problem, and 
could use a Genetic Program (GP) for the second. GAs are good for producing
models for an evaluation function, while GPs are good at actually finding
out what that evaluation function should be. The models from the GA would show
a representation of the actual inputs and outputs in a human biomechanical
system, while a GP would show what actual algorithm represents the system.

Although McGowan does not talk about optimization methods in his paper, the
research uses Simulated Annealing. This is a method used for steady convergence
to an optimal solution. This is most likely a local solution in the full domain
of potential models. What all optimizations want is to find better global 
solutions, instead of getting stuck on good local solutions. GAs are better
at this that methods like Simulated Annealing, if they are set up correctly.

Other reseachers also use Simulated Annealing, but a paper by WHO? \cite{who}
tries to paralellize the Simulated Annealing method. Parallelization is a 
method of programming that finds tasks that can be done simultaneously, and
does them all at once on many CPUs / GPUs. The proposed project will use GAs
that behave simularly to Simulated Annealing at first, and then more variance
will be tried. Through the proposed project, an eye for parallelizable areas
will also be active. If any areas are paralellizable, the gains will be 
analized and produced. If there is time, parallelization may even be 
implemented.

%\required{Intellectual Merit}
% This is why your project is interesting and will help further
% knowledge in the field of mathematics. 

%\required{Broader Impacts}
% There are 4 kinds of broader impacts.
% 1. advance discovery and understanding while promoting teaching,
% training and learning
% 2. broaden the participation of underrepresented groups
% 3. disseminated broadly to enhance scientific and technological
% understanding
% 4. benefits of the proposed activity to society

