
%%%%%%%%% MASTER -- compiles the 4 sections

\documentclass[11pt,letterpaper]{article}

\usepackage{graphicx}
\usepackage{verbatim}
\usepackage{listings}
\usepackage{amssymb,amsmath}

%%%%%%%%%%%%%%%%%%%%%%%%%%%%%%%%%%%%%%%%%%%%%%%%%%%%%%%%%%%%%%%%%%%%%%%%%
\pagestyle{plain}                                                      %%
%%%%%%%%%% EXACT 1in MARGINS %%%%%%%                                   %%
\setlength{\textwidth}{6.5in}     %%                                   %%
\setlength{\oddsidemargin}{0in}   %% (It is recommended that you       %%
\setlength{\evensidemargin}{0in}  %%  not change these parameters,     %%
\setlength{\textheight}{8.5in}    %%  at the risk of having your       %%
\setlength{\topmargin}{0in}       %%  proposal dismissed on the basis  %%
\setlength{\headheight}{0in}      %%  of incorrect formatting!!!)      %%
\setlength{\headsep}{0in}         %%                                   %%
\setlength{\footskip}{.5in}       %%                                   %%
%%%%%%%%%%%%%%%%%%%%%%%%%%%%%%%%%%%%                                   %%
\newcommand{\required}[1]{\section*{\hfil #1\hfil}}                    %%
\renewcommand{\refname}{\hfil References Cited\hfil}                   %%
\bibliographystyle{plain}                                              %%
%%%%%%%%%%%%%%%%%%%%%%%%%%%%%%%%%%%%%%%%%%%%%%%%%%%%%%%%%%%%%%%%%%%%%%%%%

%PUT YOUR MACROS HERE

\date{Due March 27th, 2012}
\title{CS 395 Homework 6}

\author{Colby Blair}

\begin{document}
\maketitle

\begin{center}

Grade: \_\_\_\_\_\_\_\_\_\_\_\_\_\_\_\_\_\_\_\_
\end{center}

\thispagestyle{empty}

\pagebreak


\section*{PROBLEMS}

\subsection*{1.}
First, the matrix product using Strassen's algorithm is as follows:
\begin{eqnarray}
	A = 	\bigl(
		\begin{smallmatrix}
		A_{11} & A_{12} \\
		A_{21} & A_{22}
		\end{smallmatrix} \bigr)
	B = 	\bigl(
		\begin{smallmatrix}
		B_{11} & B_{12} \\
		B_{21} & B_{22}
		\end{smallmatrix} \bigr)
\\
	A = 	\bigl(
		\begin{smallmatrix}
		1 & 3 \\
		7 & 5
		\end{smallmatrix} \bigr)
	B = 	\bigl(
		\begin{smallmatrix}
		6 & 8\\
		4 & 2
		\end{smallmatrix} \bigr)
\end{eqnarray}

We then find the $ \frac{n}{2} * \frac{n}{2} $ matrices:
\begin{eqnarray}
	S_1 = B_{12} - B_{22} = 8 - 2 = 6 \\
	S_2 = A_{11} + A_{12} = 1 + 3 = 4 \\
	S_3 = A_{21} + A_{22} = 7 + 5 = 12 \\
	S_4 = B_{21} - B_{11} = 4 - 6 = -2 \\
	S_5 = A_{11} + A_{22} = 1 + 5 = 6 \\
	S_6 = B_{11} + B_{22} = 6 + 2 = 8 \\
	S_7 = A_{12} - A{21} = 3 - 5 = -2 \\
	S_8 = B_{21} + B_{22} = 4 + 2 = 6 \\
	S_9 = A_{11} - A_{21} = 1 - 7 = -6 \\
	S_{10} = B_{11} + B_{12} = 6 + 8 = 14
\end{eqnarray}

We then compute the product of A and B matrices as follows:
\begin{eqnarray}
	P_1 = A_{11} * S_1 = 1 * 6 = 6 \\
	P_2 = S_2 * B_{22} = 4 * 2 = 8 \\
	P_3 = S_3 * B_{11} = 12 * 6 = 72 \\
	P_4 = A_{22} * S_4 = 5 * (-2) = -10 \\
	P_5 = S_5 * S_6 = 6 * 8 = 48 \\
	P_6 = S_7 * S_8 = -2 * 6 = -12 \\
	P_7 = S_9 * S_{10} = (-16) * 14 \\
	= -84
\end{eqnarray}

The $ \frac{n}{2} * \frac{n}{2} $ sub matrices of the product:
\begin{eqnarray}
	C_{11} = P_5 + P_4 - P_2 + P_6 \\
	= 48 - 10 - 8 - 12 \\
	= 18 \\
	C_{12} = P_1 + P_2 \\
	= 6 + 8 \\
	= 14 \\
	C_{21} = P_3 + P_4 \\
	= 72 - 10 \\
	= 62 \\
	C_{22} = P_5 + P_1 - P_3 - P_7 \\
	= 48 + 6 - 72 + 84 \\
	= 66
\end{eqnarray}

The resulting matrix for $C$ is:
\begin{eqnarray}
	C = \bigl(
		\begin{smallmatrix}
		C_{11} & C_{12} \\
		C_{21} & C_{22}
		\end{smallmatrix} \bigr)
\\
	C = \bigl(
		\begin{smallmatrix}
		18 & 14 \\
		62 & 66
		\end{smallmatrix} \bigr)
\end{eqnarray}

\subsection*{2.}
\begin{lstlisting}
						
STRASSENS_ALG(A,B)
\end{lstlisting}

\lstset{numbers=left}
\begin{lstlisting}
n = rows in A

A = matrix with n/2 x n/2 dimensions
B = matrix with n/2 x n/2 dimensions
C = matrix with n/2 x n/2 dimensions

if == 2
	S1 = B12 - B22
	S2 = A11 + A12
	S3 = A21 + A22
	S4 = B21 - B11
	S5 = A11 + A22
	S6 = B11 + B22
	S7 = A12 - A22
	S8 = B21 + B22
	S9 = A11 + A21
	S10 = B11 + B12

	P1 = A11 * S1
	P2 = S2 * B22
	P3 = S3 * B11
	P4 = A22 * S4
	P5 = S5 * S6
	P6 = S7 * S8
	P7 = S9 * S10
	C11 = P5 + P4 - P2 + P6
	C12 = P1 + P2
	C21 = P3 + P4
	C22 = P5 + P1 - P3 - P7

return C
\end{lstlisting}

\subsection*{3.}
Consider the recurrence for each:
\\
\begin{tabular}{l l}
	$ T(n) $	&	$ = 132464 T(n/68) + n^2 $ \\
			&	$ = \Theta( n^{lg_{68} 132464} ) $ \\
			&	$ \approx \Theta( n^{2.795128} ) $
\end{tabular}

\begin{tabular}{l l}
	$ T(n) $	&	$ = 143640 T(n/70) + n^2 $ \\
			&	$ = \Theta( n^{lg_{70} 143640} ) $ \\
			&	$ \approx \Theta( n^{2.795128} ) $
\end{tabular}

\begin{tabular}{l l}
	$ T(n) $	&	$ = 155424 T(n/72) + n^2 $ \\
			&	$ = \Theta( n^{lg_{72} 155424} ) $ \\
			&	$ = \Theta( n^{2.795147} ) $
\end{tabular}

Strassen's algorithm runs in $\Theta( n^{lg 7} ) \approx \Theta( n^{2.81} ) $. All of the above outperform 
Strassen's algorithm.


\subsection*{4.}
Let $ (a + bi) (c + di) = (ac - bd) + (ad + bc) i$.

Now consider the following:
\begin{eqnarray}
	p_1 = (a + b) * (c + d) \\
	p_2 = a * c \\
	p_3 = b * d \\
	ac - bd = p_2 - p_3 \\
	ad + bc = p_1 = p_2 - p_3
\end{eqnarray}

Using only $ p_1, p_2, $ and $ p_3 $, the product of two complex numbers can be computed.


\subsection*{5.}
Using the substitution method, we will prove that $T(n) = T(n -1) + n $ is $O(n^2)$.

\subsubsection*{Assumption}
$T(n) = T(n-1) + n$

\subsubsection*{Substitution} 
\begin{eqnarray}
	T(n) = T(n-1) + n \\
	T(n-1) = T(n-2) + (n-1) \\
	T(n-2) = T(n-3) + (n-2) \\
	...
\end{eqnarray}

Adding up the equations as follows:
\begin{eqnarray}
	T(n-1) + T(n-2) + ... T(2) + T(1) + n + (n-1) + n(-2) + ... 3 + 2 +1 \\
	T(n) = n + (n-1) + (n-2) + ... 2 + 1 \\
	= \frac{n(n+1)}{2} \\
	= O(n^2)
\end{eqnarray}

Therefore, we have shown that $T(n) = O(n^2)$.


\subsection*{6.}
We will show that $ T(n) = 4 T( \frac{n}{3} ) + n $ is $ T(n) = \Theta(n^{log_{3}4 } ) $. Consider 
$ a = 4, b = 3$. $n^{log_b a} = n^{log_3 4} = \Theta(n^2), f(n) = O(n^{log_3 4 - \epsilon} )$, 
when $\epsilon = 1$. Applying case 1 of the Master Theorem, we conclude that the solution is 
$ T(n) = \Theta(n^2) $.

Using the \textbf{Substitution Method}, we assume with the induction hypothesis that 
$T(n/3) \le c(\frac{n}{3})^2 - c(\frac{n}{3})$. Consider:
\begin{eqnarray}
	T(n) = 4T(\frac{n}{3}) + n 
\end{eqnarray}

\end{document}