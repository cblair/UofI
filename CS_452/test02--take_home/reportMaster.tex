
%%%%%%%%% MASTER -- compiles the 4 sections

\documentclass[11pt,letterpaper]{article}

\usepackage{graphicx}
\usepackage{verbatim}
\usepackage{listings}
\usepackage{amssymb,amsmath}
\usepackage{enumerate}

%%%%%%%%%%%%%%%%%%%%%%%%%%%%%%%%%%%%%%%%%%%%%%%%%%%%%%%%%%%%%%%%%%%%%%%%%
\pagestyle{plain}                                                      %%
%%%%%%%%%% EXACT 1in MARGINS %%%%%%%                                   %%
\setlength{\textwidth}{6.5in}     %%                                   %%
\setlength{\oddsidemargin}{0in}   %% (It is recommended that you       %%
\setlength{\evensidemargin}{0in}  %%  not change these parameters,     %%
\setlength{\textheight}{8.5in}    %%  at the risk of having your       %%
\setlength{\topmargin}{0in}       %%  proposal dismissed on the basis  %%
\setlength{\headheight}{0in}      %%  of incorrect formatting!!!)      %%
\setlength{\headsep}{0in}         %%                                   %%
\setlength{\footskip}{.5in}       %%                                   %%
%%%%%%%%%%%%%%%%%%%%%%%%%%%%%%%%%%%%                                   %%
\newcommand{\required}[1]{\section*{\hfil #1\hfil}}                    %%
\renewcommand{\refname}{\hfil References Cited\hfil}                   %%
\bibliographystyle{plain}                                              %%
%%%%%%%%%%%%%%%%%%%%%%%%%%%%%%%%%%%%%%%%%%%%%%%%%%%%%%%%%%%%%%%%%%%%%%%%%

%PUT YOUR MACROS HERE

\date{Due April 4th, 2012}
\title{CS 452 Second Exam}

\author{Colby Blair}

\begin{document}
\maketitle

\begin{center}

Grade: \_\_\_\_\_\_\_\_\_\_\_\_\_\_\_\_\_\_\_\_
\end{center}

\thispagestyle{empty}

\pagebreak


\section{}

\textbf{a.} F \\
\textbf{b.} T \\
\textbf{c.} T \\
\textbf{d.} F \\
\textbf{e.} T \\


\section{}
\begin{enumerate}[(a)]
	\item Variables allocated in the \textbf{bss} section.
		\begin{itemize}
			\item In languages like C, the compiler typicall adds code to the beginning (in our case, to the
				program memory) that initializes variables found in the bss section to 0, NULL, etc.
				Variables found in the bss section are typically globals or static variables, seen through
				the life of the program.
		\end{itemize}
	\item Variables allocated in the \textbf{data} section
		\begin{itemize}
			\item Variables in the data section are initialized variables. For Harvard archetictures like ours,
				each time the variable is referenced, a load in program memory must be inserted to
				bind the variable to a register. Each time the variable is assigned to, a store must be
				inserted into program memory 
		\end{itemize}
	\item Local variables declared (but not initialized) as static
		\begin{itemize}
			\item At the begging of program memory, code is inserted to initialize static variables to 0,
				NULL, etc.
		\end{itemize}
	\item an interrupt service routine
		\begin{itemize}
			\item Enabling interrupts will cause the compiler to create an interrupt vector/table at
				the beginning of program memory. The values in this table are usually the memory
				addresses for the respective interrupt handler functions.
		\end{itemize}
\end{enumerate}


\section{}
The \textit{round-robin} for our RTOS is a \textbf{non-preemptive} scheduling algorithm. It is the simplest
possible algorithm for running multiple tasks. It calls the first task, lets it run until it finishes, then calls the
second, etc. Since tasks are essentially unmanaged, \textbf{starvation} is possible, and tasks with 
higher prority can end up waiting on lower priority tasks.

By comparison, \textit{multi-user round-robin} algorithms preemptively give each task even time slices. 
If a task is not finished, it is interrupted and will resume on the next round. It is a starvation free algorithm,
but it also has no task priorities.


\section{}
The primary point of the tick counter in the RTOS is to create a adjustable time unit in which to base
preemptive actions (interrupts, context switches) on. 


\section{}
Having a \textbf{fast time tick} means that context switctes happen fast. A a pro, \textbf{Higher 
priority tasks wait less} for the next time tick, in which case they get to switch with a lower priority tasks and
run. As a con, context switches can happen more often, which means the CPU has much more overhead
doing context switch operations, and less actual task execution is done.

Having a \textbf{slow time tick} means just the opposite. As a pro, there are \textbf{less context switches 
and thus less overhead}. As a con, \textbf{higher priority tasks} have to wait longer before they can switch
with a lower priority task.


\section{}
Disadvantages to disabling interrupts while tasks are in critical section:
\begin{itemize}
	\item If a task goes into an infinite loop (i.e. logic issues or waiting on IO), than the entire operating
		system is frozen.
	\item If the user forgets to renable interrupts, than the operating system may be frozen, or scheduling
		will not function properly.
\end{itemize}


\section{}
\scriptsize
\begin{lstlisting}
int SEMAPHORE1 = TRUE; /*available*/

int CHECK_SEMAPHORE_INTERVAL = 3; /*seconds*/

/*
 * This function will make the callee wait until the semaphore is ready, 
 * and then reserves the semaphore for the callee. The callee must free 
 * it when it is done with its critical section.
 */
void P(int semaphore)
{
	while(semaphore != TRUE)
	{
		delay(CHECK_SEMAPHORE_INTERVAL);
	}

	/*
	  * we execute the semaphore reservation as quickly as possible
	  * after it is freed, to avoid any mutex issues by another task 
	  * doing the same thing.
	  */
	
	semaphore = FALSE; /*reserved*/

	return;
}
\end{lstlisting}
\normalsize


\pagebreak


\section{}
Things that UIK has to preserve during a context switch are:
\begin{itemize}
	\item The Stack Pointer register
	\item The General Purpose register
	\item The Status register
	\item Program Counter
\end{itemize}


\section{}
UIK Design

The task stack, PCB, and more can be seen coded as follows:
\scriptsize
\lstinputlisting{src/task_stack.h}
\normalsize

\subsection{Task Stack(s)}
I'm starting out with one task stack. It is specified in the same spot as the process control block (PCB).
I've declared it with PROGMEM, so it lives in program memory as opposed to data memory. Anticipating
lots of context switching, this will hopefully keep latency to a minimum, since it won't have to go to 
data memory for context information. The task stack is \textbf{static} in size, but the values for the 
tasks can be modified. New tasks have to be compiled in.


\subsection{Task Control Block}
Contained in my Task Control Block is all the standard stuff from our textbook and classnotes: 
(function) \textbf{pointer} to the task, \textbf{state/status} (blocked, ready, running), 
\text{pid} (process/task id), \text{program counter}, \textbf{register} (Status, General Purpose), 
and \textbf{memory limits} (upper and lower bounds).


\subsection{Stack Pointer}
To access the Stack pointer, we'll have to write some assembly code. I choose to use inline assembly in C.

\begin{lstlisting}
   asm volatile ( 
		...
               "in      r0, 0x3d            \n\t"   \ 
               "st      x+, r0               \n\t"   \ 
               "in      r0, 0x3e            \n\t"   \ 
               "st      x+, r0               \n\t"   \
		);
\end{lstlisting}

The Stack Pointer is now in the General Purpose Register, and I can access this way. 0x03d and 0x3E above
are  the locations of the Stack Pointer in memory. I save this later in a variable in my C code (see code 
snippet above).

\subsection{Context}
I save the context of each task (function) in there respective PCB's. The Task Stack and the PCB's are
defined in program memory using the AVR PROGMEM macro.

\subsection{Program counter register}
Saving the Program Counter is tricky, apparently. You have to do some sort of jump, and the PC value is 
popped on top of the stack. In assembly, an \textbf{RCALL} to a relative position of 0. We can then store 
the value to whatever we like.

To restore the program to the PC, do an assembly ICALL (immediate jump) to our stored PC value.

We can use inline assembly in the C source to do all this. Example:
\begin{lstlisting}
asm volatile("RCALL 0\n\t"
             "nop\n\t"
	      ...
             "nop\n\t"
             "ICALL r1\n\t"
             ::);
\end{lstlisting}


\end{document}