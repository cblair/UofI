
%%%%%%%%% SUMMARY -- 1 page, third person
% e.g:  "The PI will prove" not "I will prove"

%Introduction / Summary (1 page).  The following information is typically included in proposal introductions for this 
%grant. The summary should lead to the proposal body but it should also be a self-contained document, 
%summarizing what is being proposed and why.   
 
% Provide appropriate background that sets the context for the problem and the objectives. 
%State the specific problem (may also be thought of as a need, central research question, or hypothesis) 
%your research will address. 
% State the objectives of the proposed research. 
%State any limits of the proposed research  (may not be needed). 
%Summarize the research methods you will use to achieve research objectives (if clearly covered elsewhere 
%in the introduction, a separate section may not be needed). 


%\required{Project Summary}
\section{Project Summary}
% This should be a brief statement of the problem you plan to address.
% It should look something like an abstract. 

\subsection{Background}

In the areas of ecology and bioinformatics, there is an explosion of data. The
amount of bioinformatics data in research institutions doubles every 9 months 
\cite{james}. Gains in processing power, however, are decreasing, as the end
of Moore's Law's approaches \cite{gordon_moore}. As processor speed gains decrease, 
a vacuum for high performance computing is growing. More and more data has 
to be ignored. Data that was expensive to collect.

In spite of the growth of data and platue of processor computing power, a 
solution exists that allows scientists and mathematicians to conduct their 
research. High Performance Computing (HPC) networks groups of computers 
together as one supercomputer. HPC then uses parallel programs that do many 
of their tasks in parallel, or at the same time. This reduces run time. 
The result is a resource that allows researchers to do years worth of 
processing in months.

\subsection{Problem Statement}

Unfortunately, computing is typically an afterthought in most research projects.
Most researchers focus on data collection, and assume that the power to
process their data is cheap and widely available. This assumption leads to 
overcollection of data, and almost no relative analysis of that data. Without
consultation of computer scientists, most scientists' research is a fraction of
what it should be.

\subsection{Objectives}

The proposal here is to conduct research on existing algorithms used in 
CNR, and to parallelize them on a HPC cluster. Algorithms will be take from 
recently completed and ongoing research projects, and they will be demonstrated
in an HPC environment. Concepts of HPC and Parallel Programming will be covered,
and their implementation in current research will be shown.

Demonstrating HPC and Parallel Programming methods in research could be very 
beneficial to fellow researchers. It will create interest in computer science,
and encourage a much needed interdisciplinary dialogue. More project proposals 
would include more realistic processing projections. Less research data would be
thrown away, and overall projects' success would rise.

The objective of this project proposal is to establish a computing cornerstone
in the College of Natural Resources. It is to educate scientists and 
mathematicians with HPC concepts. The objective is to spread experience with 
computing on a big scale, so that scientists can shape their research around
what is achievalbe. The objective will also affect how mathematicians develop 
their algorithms.

Limitations to the project are drawn at the operating system level. Some detail
is given on the HPC Operating System, but building one is beyond the scope of 
the research. Operating systems will only be discussed as much as is needed
to allow readers to work the software. The rest of the operating system 
documentation should be enough.

%\required{Intellectual Merit}
% This is why your project is interesting and will help further
% knowledge in the field of mathematics. 

%\required{Broader Impacts}
% There are 4 kinds of broader impacts.
% 1. advance discovery and understanding while promoting teaching,
% training and learning
% 2. broaden the participation of underrepresented groups
% 3. disseminated broadly to enhance scientific and technological
% understanding
% 4. benefits of the proposed activity to society

