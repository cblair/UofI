
%%%%%%%%% MASTER -- compiles the 4 sections

\documentclass[11pt,letterpaper]{article}

\usepackage{graphicx}
\usepackage{verbatim}
\usepackage{listings}
\usepackage{amssymb,amsmath}

%%%%%%%%%%%%%%%%%%%%%%%%%%%%%%%%%%%%%%%%%%%%%%%%%%%%%%%%%%%%%%%%%%%%%%%%%
\pagestyle{plain}                                                      %%
%%%%%%%%%% EXACT 1in MARGINS %%%%%%%                                   %%
\setlength{\textwidth}{6.5in}     %%                                   %%
\setlength{\oddsidemargin}{0in}   %% (It is recommended that you       %%
\setlength{\evensidemargin}{0in}  %%  not change these parameters,     %%
\setlength{\textheight}{8.5in}    %%  at the risk of having your       %%
\setlength{\topmargin}{0in}       %%  proposal dismissed on the basis  %%
\setlength{\headheight}{0in}      %%  of incorrect formatting!!!)      %%
\setlength{\headsep}{0in}         %%                                   %%
\setlength{\footskip}{.5in}       %%                                   %%
%%%%%%%%%%%%%%%%%%%%%%%%%%%%%%%%%%%%                                   %%
\newcommand{\required}[1]{\section*{\hfil #1\hfil}}                    %%
\renewcommand{\refname}{\hfil References Cited\hfil}                   %%
\bibliographystyle{plain}                                              %%
%%%%%%%%%%%%%%%%%%%%%%%%%%%%%%%%%%%%%%%%%%%%%%%%%%%%%%%%%%%%%%%%%%%%%%%%%

%PUT YOUR MACROS HERE

\date{Due May 7th, 2012}
\title{CS 395 Homework 10}

\author{Colby Blair}

\begin{document}
\maketitle

\begin{center}

Grade: \_\_\_\_\_\_\_\_\_\_\_\_\_\_\_\_\_\_\_\_
\end{center}

\thispagestyle{empty}

\pagebreak


\section*{PROBLEMS}

\subsection*{1. 7.1-1}
Consider the array $ A = \{13,19,9,5,12,8,7,4,21,2,6,11\} $. The PARTITION(A) algorithm operations are 
as follows:

\subsubsection*{a).}
\begin{tabular}{ | c | c | c | c | c | c | c | c | c | c | c | c | }
		p,j	&		&		&		&		&		&		&		&		&		&		&	r	\\
	\hline
		13	&	19	&	9	&	5	&	12	&	8	&	7	&	4	&	21	&	2	&	6	&	11	\\
	\hline
\end{tabular} \\

\subsubsection*{b).}
\begin{tabular}{ | c | c | c | c | c | c | c | c | c | c | c | c | }
		p,i	&	j	&		&		&		&		&		&		&		&		&		&	r	\\
	\hline
		13	&	19	&	9	&	5	&	12	&	8	&	7	&	4	&	21	&	2	&	6	&	11	\\
	\hline
\end{tabular}

\subsubsection*{c).}
\begin{tabular}{ | c | c | c | c | c | c | c | c | c | c | c | c | }
		p,i	&		&	j	&		&		&		&		&		&		&		&		&	r	\\
	\hline
		13	&	19	&	9	&	5	&	12	&	8	&	7	&	4	&	21	&	2	&	6	&	11	\\
	\hline
\end{tabular}

\subsubsection*{d).}
\begin{tabular}{ | c | c | c | c | c | c | c | c | c | c | c | c | }
		i	&	p	&		&	j	&		&		&		&		&		&		&		&	r	\\
	\hline
		9	&	19	&	13	&	5	&	12	&	8	&	7	&	4	&	21	&	2	&	6	&	11	\\
	\hline
\end{tabular}

\subsubsection*{e).}
\begin{tabular}{ | c | c | c | c | c | c | c | c | c | c | c | c | }
			&	i	&	p	&	j	&		&		&		&		&		&		&		&	r	\\
	\hline
		9	&	5	&	13	&	19	&	12	&	8	&	7	&	4	&	21	&	2	&	6	&	11	\\
	\hline
\end{tabular}

\subsubsection*{f).}
\begin{tabular}{ | c | c | c | c | c | c | c | c | c | c | c | c | }
		p	&	i	&	p	&		&		&	j	&		&		&		&		&		&	r	\\
	\hline
		9	&	5	&	13	&	19	&	12	&	8	&	7	&	4	&	21	&	2	&	6	&	11	\\
	\hline
\end{tabular}

\subsubsection*{g).}
\begin{tabular}{ | c | c | c | c | c | c | c | c | c | c | c | c | }
		p	&		&	i	&		&		&		&	j	&		&		&		&		&	r	\\
	\hline
		9	&	5	&	8	&	19	&	12	&	13	&	7	&	4	&	21	&	2	&	6	&	11	\\
	\hline
\end{tabular}

\subsubsection*{h).}
\begin{tabular}{ | c | c | c | c | c | c | c | c | c | c | c | c | }
		p	&		&		&	i	&		&		&		&	j	&		&		&		&	r	\\
	\hline
		9	&	5	&	8	&	7	&	12	&	13	&	19	&	4	&	21	&	2	&	6	&	11	\\
	\hline
\end{tabular}

\subsubsection*{i).}
\begin{tabular}{ | c | c | c | c | c | c | c | c | c | c | c | c | }
		p	&		&		&		&	i	&		&		&	j	&		&		&		&	r	\\
	\hline
		9	&	5	&	8	&	7	&	4	&	13	&	19	&	12	&	21	&	2	&	6	&	11	\\
	\hline
\end{tabular}

\subsubsection*{j).}
\begin{tabular}{ | c | c | c | c | c | c | c | c | c | c | c | c | }
		p	&		&		&		&	i	&		&		&		&	j	&		&		&	r	\\
	\hline
		9	&	5	&	8	&	7	&	4	&	13	&	19	&	12	&	21	&	2	&	6	&	11	\\
	\hline
\end{tabular}

\subsubsection*{k).}
\begin{tabular}{ | c | c | c | c | c | c | c | c | c | c | c | c | }
		p	&		&		&		&		&	i	&		&		&		&	j	&		&	r	\\
	\hline
		9	&	5	&	8	&	7	&	4	&	2	&	19	&	12	&	21	&	13	&	6	&	11	\\
	\hline
\end{tabular}

\subsubsection*{l).}
\begin{tabular}{ | c | c | c | c | c | c | c | c | c | c | c | c | }
		p	&		&		&		&		&		&	i	&		&		&		&	j	&	r	\\
	\hline
		9	&	5	&	8	&	7	&	4	&	2	&	6	&	12	&	21	&	13	&	19	&	11	\\
	\hline
\end{tabular}

\subsubsection*{m).}
\begin{tabular}{ | c | c | c | c | c | c | c | c | c | c | c | c | }
		p	&		&		&		&		&		&	i	&		&		&		&	j	&	r	\\
	\hline
		9	&	5	&	8	&	7	&	4	&	2	&	6	&	11	&	21	&	13	&	19	&	12	\\
	\hline
\end{tabular}


\subsection*{2. 7.2-2}
In Quicksort, if every element is the same, then the call to PARTITION(A) in Quicksort always returns r - 1. 
This leads to the worst case partitions, and Quicksort runtime $ = T(n) = \Theta(n^2) $ (worst case
runtime).


\subsection*{3. 7.3-1}
The randomization won't improve the worst case, it just makes the chances of hitting the worst case small.
So we don't consider it.


\subsection*{4. 7.4-1}
Considering the equation $ T(n) = max_{0 \le q \le n - 1} ( T(q) + T(n - q - 1)) + \Theta(n) $, we will show
that the running time $ T(n) = \Omega(n^2) $.

\subsubsection*{Inductive Hypothesis}
Given $ T(n) = \Omega(n^2) $, we assume:

\begin{eqnarray}
	T(n) \ge cn^2
\end{eqnarray}

Where $c$ is a constant. 


\pagebreak

Further:

\begin{eqnarray}
	T(n) \ge max_{0 \le q \le n - 1} ( q^2 + c(n - q - 1)^2 ) + \Theta(n) \\
		= (c) max_{0 \le q \le n - 1} ( q^2 + (n - q - 1)^2 ) + \Theta(n) \\
\end{eqnarray}

The minimum for $q^2 + (n - q - 1)^2$ can be found  in the range $ 0 \le q \le n - 1 $. We can then say:

\begin{eqnarray}
	max_{0 \le q \le n - 1} ( q^2 + (n - q - 1)^2 ) \le (n - 1)^2 = n^2 - 2n + 1 \\
\end{eqnarray}

Furthermore:

\begin{eqnarray}
	T(n) 	\ge cn^2 - c(2n - 1) \Theta(n) \\
		\ge cn^2
\end{eqnarray}

Therefore, runtime $ T(n) = \Omega(n^2) $.


\end{document}