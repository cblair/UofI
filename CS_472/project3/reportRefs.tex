
%%%%%%%%% REFERENCES -- no limit

% this should include only items referenced in the project description
% it is not a bibliography of related reading.

% Each reference must include the names of all authors (in the same
% sequence in which they appear in the publication), the article and 
% journal title, book title, volume number, page numbers, and year of 
% publication. If the document is available electronically, the website 
% address also should be identified

\section{Bibliography}

\begin{thebibliography}{99}
\bibitem{mcgowan} Neptune, R.R.; McGowan, C.P. 
	"Muscle contributions to whole-body sagittal plane angular momentum during walking" 
	{ \em Journal of Biomechanics, 2011 }. 44 6–12 : Print.
\bibitem{goffe-sa} Goffe, L.G; Ferrier, G.D.; Rogers, J. 
	"Global optimization of statistical function with simulated annealing"
	{ \em Journal of Econometrics, 1994 }. 60 65-99 : Print.
\bibitem{mcgowan-fit} Neptune, R.R.; McGowan, C.P.; Kautz, S.A.
	"Forward Dynamics Simulations Provide Insight Into Muscle Mechanical Work During Human Locomotion"
	{ \em Exerc. Sport Sci. Rev., 2009 }. Vol. 37 No. 4 203-210 : Print
\bibitem{amdahl} Amdahl, Gene (1967). "Validity of the Single Processor Approach to Achieving Large-Scale Computing Capabilities". {\em AFIPS Conference Proceedings} (30): 483–485. Print.
%\bibitem{thelen} Thelen, D.G.; Anderson, F.C. "Using computed muscle control to generate forward dynamic simulations of human walking from experimental data"
%	{ \em Journal of Biomechanics, 2006 }. 39 1107 - 1115: Print.
%\bibitem{lloyd} Lloyd, G.L.; Besier, T. F.
%	"An EMG-driven musculoskeletal model to estimate muscle forces and knee joint moments in vivo"
%	{ \em  Journal of Biomechanics, 2003 }. 36 765-776: Print.
\end{thebibliography}
