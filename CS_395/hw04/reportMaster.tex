
%%%%%%%%% MASTER -- compiles the 4 sections

\documentclass[11pt,letterpaper]{article}

\usepackage{graphicx}
\usepackage{verbatim}
\usepackage{listings}

%%%%%%%%%%%%%%%%%%%%%%%%%%%%%%%%%%%%%%%%%%%%%%%%%%%%%%%%%%%%%%%%%%%%%%%%%
\pagestyle{plain}                                                      %%
%%%%%%%%%% EXACT 1in MARGINS %%%%%%%                                   %%
\setlength{\textwidth}{6.5in}     %%                                   %%
\setlength{\oddsidemargin}{0in}   %% (It is recommended that you       %%
\setlength{\evensidemargin}{0in}  %%  not change these parameters,     %%
\setlength{\textheight}{8.5in}    %%  at the risk of having your       %%
\setlength{\topmargin}{0in}       %%  proposal dismissed on the basis  %%
\setlength{\headheight}{0in}      %%  of incorrect formatting!!!)      %%
\setlength{\headsep}{0in}         %%                                   %%
\setlength{\footskip}{.5in}       %%                                   %%
%%%%%%%%%%%%%%%%%%%%%%%%%%%%%%%%%%%%                                   %%
\newcommand{\required}[1]{\section*{\hfil #1\hfil}}                    %%
\renewcommand{\refname}{\hfil References Cited\hfil}                   %%
\bibliographystyle{plain}                                              %%
%%%%%%%%%%%%%%%%%%%%%%%%%%%%%%%%%%%%%%%%%%%%%%%%%%%%%%%%%%%%%%%%%%%%%%%%%

%PUT YOUR MACROS HERE

\date{Due February 15th, 2012}
\title{CS 395 Homework 4}

\author{Colby Blair}

\begin{document}
\maketitle

\begin{center}

Grade: \_\_\_\_\_\_\_\_\_\_\_\_\_\_\_\_\_\_\_\_
\end{center}

\thispagestyle{empty}

\pagebreak


\section*{PROBLEMS}

\subsection*{1.}
The basic definition for $\Theta$-notation states:

\small
\begin{eqnarray}
\Theta(g(n)) =		& \{f(n) : \exists c_1, c_2, n_0; 0 \leq c_1g(n) \leq f(n) \leq c_2 g(n)  \forall  n \geq n_0 \} 	
					& \Theta\mbox{-notation Definition} \label{eq:big_theta} \\
O(g(n)) =			& \{f(n): f(n) \leq g(n) \} 
					& O(n)\mbox{-notation Definition} \\
\Omega(g(n)) = 	& \{f(n): f(n) \geq g(n) \}
					& \Omega\mbox{-notation Definition} %\\
\end{eqnarray}
\normalsize

Note that Theorem 3.1 states: For any two functions $f(n)$ and $g(n)$, we have $f(n) = \Theta(g(n))$ if and
 only if $f(n) = O(g(n)$ and $f(n) = \Omega(g(n))$. Let's assume this is true:

\begin{eqnarray}
\Omega(g(n)) = f(n) = O(g(n)) 
				& \\
f(n) \geq g(n) = f(n) = f(n) \leq g(n) 
				& \mbox{Substitution} \\
f(n) = g(n) = f(n) = f(n) = g(n) 
				& \mbox{Equivalence} \label{eq:equiv}
\end{eqnarray}

Step \ref{eq:equiv} above matches the conditions for $\Theta$-notation described in step \ref{eq:big_theta}.
But what about for when $f(n) \neq g(n)$? Assume an equation where this is true:

\begin{eqnarray}
\Omega(g(n)) < f(n) < O(g(n)) 
				& \label{eq:contra_hypo}\\
f(n) \geq g(n) < f(n) < f(n) \leq g(n) 
				& \mbox{Substitution} \\
f(n) = g(n) \neq f(n) \neq f(n) = g(n) 
				& \mbox{Contradiction} \label{eq:contra}
\end{eqnarray}

Due to the contradiction assumption in step \ref{eq:contra_hypo}, the statement is simplified to a statement
in step \ref{eq:contra} that is a contradiction. 

Therefore, by Proof by Contradiction, Theorem 3.1 is valid.


\subsection*{2}
Problem two states that $f(n)$ and $g(n)$ are monotonically increasing functions. By definition, this means:

\begin{figure}[!h]

	\begin{center}
	$ m \leq n \Rightarrow f(m) \leq f(n). $
	\end{center}

\caption{Definition for Monotonicity, Introduction to Algorithms Chapter 3.2}
\label{monotonicity}
\end{figure}

This means for both $f(n)$ and $g(n)$:
\begin{eqnarray}
	m \leq n \Rightarrow f(m) \leq f(n). \\
	m \leq n \Rightarrow g(m) \leq g(n).
\end{eqnarray}

\textbf{First}, consider the function $h(n) = f(n) + g(n)$. Since both $f()$ and $g()$ are both monotonic, 
$f(m) + g(m) \leq f(n) + g(n). $ Therefore, $m \leq n \Rightarrow h(m) \leq h(n) $, and the function
$h(n) = f(n) + g(n)$ by defintion is monotonic.

\textbf{Second}, consider the function $h(n) = f(g(n))$. Once can simply represent $g(m)$ as some
variable $m_1$, and $g(n)$ as some variable $n_1$. Therefore, $h(n) = f(n_1)$ and $h(m) = f(m_1)$.
Since $g()$ is monotonic, then $m_1 \leq n_1$. 

Since both $f()$ and $g()$ are both monotonic, and $m_1 \leq n_1 $, 
$f(m_1) \leq f(n_1). $ Therefore, $m \leq n \Rightarrow h(m) \leq h(n) $, and the function
$h(n) = f(g(n))$ by defintion is monotonic.

\textbf{Third}, consider the equation $h(n) = f(n) * g(n) \exists f(n) + g(n) \geq 0 $. If $f(n)$ and $g(n)$ are
monotonic, then $f(m) \leq f(n)$ and $g(m) \leq  g(n)$. Therefore, $f(m) * g(m) \leq f(n) * g(n)$. Substituting 
$(h()$, $h(m) \leq h(n)$, and by definition, is monotonic.


\subsection*{3}
Proof that $a^{log_b(c)} = c^{log_b(a)}$:
\begin{eqnarray}
	a^{ \frac{ log_a(c) }{ log_a(b) } } = c^{ \frac{ log_c(a) }{ log_c(b) } }		& \mbox{Changing bases rule}\\
	a^{ log_a(c) * log_b(a) } = c^{ log_c(a) * log_b(c) } 					& \mbox{Logarithm property} \\
	c * a^{ log_b(a) } = a * c^{ log_b(c) }							& \mbox{Logarithm property} %\\
\end{eqnarray}
This is as far as I got, did not find properties to continue solving this.

\subsection*{4}
Consider the equation $x^2 = x + 1$. The equation can be reduced as follows:
\begin{eqnarray}
	0 = 		& 	x^2 - x - 1 								&	\\
	x = 		&	\frac{ -1 \pm \sqrt{ 1^2 - 4 * (1) * (-1) } }{2 * 1}	& 	\mbox{Quadratic Formula} \\
	x = 		&	\frac{ 1 \pm \sqrt{5} }{2}						&	\\
	x =		&	\{ \frac{ 1 - \sqrt{5} }{ 2 }, \frac{ 1 + \sqrt{5} }{ 2 } \}	\label{eq:quad}
\end{eqnarray}

The definitions for $\phi$ and $\wedge\phi$ are:
\begin{eqnarray}
	\phi = 		&	\frac{ 1 + \sqrt{5} }{ 2 } 	\label{eq:phi}\\
	\wedge\phi =	&	\frac{ 1 - \sqrt{5} }{ 2 }	\label{eq:conj_phi}
\end{eqnarray}

The set $ \{ \phi, \wedge\phi \} = \{ \frac{ 1 - \sqrt{5} }{ 2 }, \frac{ 1 + \sqrt{5} }{ 2 } \} = $ the solution set
for $x^2 = x + 1$ as shown in step \ref{eq:quad}. Therefore, $\phi$ and $\wedge\phi$ satisfy $x^2 = x + 1$.


\subsection*{5}
For this problem, it will be shown by Proof by Induction that the $i$th Fibonacci umber satisfies the equality:
\begin{eqnarray}
	F_i 		& = \frac{ \phi^i - \wedge\phi^i }{ \sqrt{5} } 
\end{eqnarray}

\subsubsection*{Step 1: Base step}
Let $i = 2$. It can then be shown:

\Large
\begin{eqnarray}
	F_2	=	& \frac{ \phi^2 - \wedge\phi^2 }{ \sqrt{5} } 			
				&	\\
	F_2 =	& \frac{ ( \frac{ 1 + \sqrt{5} }{ 2 } )^2 - ( \frac{ 1 - \sqrt{5} }{ 2 } )^2 }{ \sqrt{5} } 
				& \mbox{Definitions from steps \ref{eq:phi} and \ref{eq:conj_phi} } \\
	F_2 =	& \frac{ ( \frac{ 1 + \sqrt{5} }{ 2 } )^2 - ( \frac{ 1 - \sqrt{5} }{ 2 } )^2 }{ \sqrt{5} } 
				& 	\\
	F_2 =	& \frac{ 
				( \frac{ 1 + \sqrt{5} }{ 2 } * \frac{ 1 + \sqrt{5} }{ 2 }) - 
				( \frac{ 1 - \sqrt{5} }{ 2 } * \frac{ 1 - \sqrt{5} }{ 2 })
				}
				{ \sqrt{5} } 
				& 	\\
	F_2	=	& \frac{ \frac{1 + 2\sqrt{5} + 5}{4} - \frac{1 - 2\sqrt{5} + 5}{4} }
				{ \sqrt{5} } 
				& \\
	F_2 =	& \frac{ \frac{ 1 + 2\sqrt{5} + 5 - 1 + 2\sqrt{5} - 5 }{ 4 } }
				{ \sqrt{5} }
				& \\
	F_2 =	& \frac{ \frac{ 4\sqrt{5} }{ 4 } }
				{ \sqrt{5} }
				& \\
	F_ 2 =	& \frac{\sqrt{5}}{\sqrt{5}}
				& \\
	F_2 = 	& 1
				& 
\end{eqnarray}
\normalsize

Note that the \textbf{Fibonacci numbers} are defined by the following recurrence:
\begin{eqnarray}
	F_0 = 	& 0, \\
	F_1 = 	& 1, \\
	F_i =	& F_{i-1} + F_{i-2} \label{eq:fib_nums1}
\end{eqnarray}

Per the definition of the \textbf{Fibonacci numbers}, the 2nd number in the sequence is 1, so the Base Step
holds.

\subsubsection*{Induction Assumption}
For the Induction Assumption, it is assumed that $ F_i = \frac{ \phi^i - \wedge\phi^i }{ \sqrt{5} } $ for $i$.

\subsubsection*{Induction Step}
For the Induction Step, it will be shown that $ F_k = \frac{ \phi^k - \wedge\phi^k }{ \sqrt{5} } $ for $k = i + 1$.

First, we will show that $ F_k = \frac{ \phi^k - \wedge\phi^k }{ \sqrt{5} } $ for $k = i$.
\begin{eqnarray}
	F_i = 	& F_{i+1} + F_i 		
		&	\mbox{Definition from step \ref{eq:fib_nums1} } \\
	     =		& \frac{ \phi^{i-1} - \wedge\phi^{i-1} }{ \sqrt{5} } + \frac{ \phi^{i-2} - \wedge\phi^{i-2} }
			{ \sqrt{5} }
				&	\mbox{The Induction Assumption} \\
	     =		& \frac{ 
				\phi^{i-1} - \wedge\phi^{i-1} + \phi^{i-2} - \wedge\phi^{i-2} }
				{ \sqrt{5} }
				& \\
	     =		& \frac{ 
				\phi^{i-2}(\phi + 1) - \phi^{i-2}( \wedge\phi^{i} + ) }
				{ \sqrt{5} }
				& \\
	     =		& \frac{ \phi^i - \wedge\phi^i }{ \sqrt{5} }
				& \label{eq:prob_5_proof} \\
\end{eqnarray}

Therefore, as shown above leading to step \ref{eq:prob_5_proof}, the Inductions Step holds, showing proof by Induction.


\end{document}