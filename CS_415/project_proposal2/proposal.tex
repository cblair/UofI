\documentclass[12pt]{article}
\usepackage[utf8]{inputenc}
\usepackage{amsmath}
\usepackage{graphicx}
\title{CS 470 Spring 2011 \\
     Project Proposal}
\author{Colby Blair}
\date{Due May 2nd, 2011}
\begin{document}
\maketitle

\pagebreak
\setcounter{page}{1}

In the study of animal movements, there are many algorithms for calculating potential paths. One method for
calculating the probability of an animal in an area is using the Brownian Bridge Motion Model (BBMM) \cite{bb}.
The BBMM can create a probability map of an area to then predict where animals will tend to migrate. Since 
animals can move in slightly different ways, it can be beneficial to analyze each individual seperately.

The problem with the BBMM is that it only makes a probability map off of past animal movements. Since this
doesn't tell us much about preference to habitat, the Synoptic Model \cite{syn} can also be used
The Synoptic Model starts with a null model, and optimizes environment covariates for the following models 
over a maximum likelihood algorithm.

One can combine BBMM and SM. Selecting sets of locations to run the Synoptic Model on can be left to BBMM.
The results are being finalized in a research project, Northwest Cascade Mountain Goat Research Project, at
the University of Idaho College of Natural Resources. NCMGRP has found that mountain goats overwhelming 
prefer slope in terrain, and talus rock. They care little about much else. The mountain goat appears to be
a very paranoid animal, and are sensitive to predetors. So sensitive, that they stay in areas that are hard
for predators to navigate.

My proposal is to simulate a slope and talus environmet, for virtual goats to navigate. I will then simulate
natural and human hunters, and place their movement limitations as well into the environment. I will take the
resulting predictions from the study, and apply them to either StarLogo or V6 simulations, and then model
real locations. I will then run simulations to answer environmental questions, like reintroduction of goats
or predators in certain areas. I will also simulate environmental changes, and predict what goat populations 
would do. The questions will be real questions of reintroduction, and the results will calculate where future 
populations will move, and if they will survive.

\pagebreak

\section{Bibliography}

\begin{thebibliography}{99}
\bibitem{bb} Horne, Garton, et al. "Analyzing Animal Movements using Brownian 
	Bridges". { \em Ecology } 88(9) 2007: 2354-2363. Print
\bibitem{syn} Horne, Garton, Rachlow. "A synoptic model of animal space use: 
	Simultaneous estimation of home range, habitat selection, and 
	inter\/intra-species relationships" {\em Ecological Modelling} 
	214, 2008: 338-348. Print.

\end{thebibliography}

\end{document}
