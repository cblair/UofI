\section{Introduction}
Walking requires a lot of complex tasks to make locomotion possible 
\cite{mcgowan}. The inputs to muscle motor nerves and the outputs of joint
motion are one way of measuring all these complex tasks. The data collected
as inputs and outputs fits nicely into Evolutionary Computation (EC). EC uses 
many measured inputs and expected outputs, and evolves a system that will 
produce those expected outputs on its own. 

The EC program will do this with some margin of error, of course, but the 
error of the solution should be small. The EC solution will then be useful 
when there are a set of inputs that the output is not already know, like in 
a simulation. In this report, we will use a \textbf{Genetic Program (GA)} as
it fits the multiple input to one output model well. The expected output is 
used to evaluate the GA individuals using a fitness function. This fitness 
function calls the muscle simulation function with paramaters help by each
GA individual. These parameters are changed in a scheme discribed next, and
the individuals' fitnesses depend on how well their parameters do in the 
simulation.

\subsection{Hypothesis}
As noted above, the GA is set up to represent SA as closely as possible,
at least at first. Our hypothesis is that the GA will find better solutions
than the SA, while having the advantage of trying more diverse solution that
would could cause worse solutions for an SA. The GA will be able to better
search the domain, without as many problem with local optimums. These 
advantages may come at the cost of process time, but are hypothesized to 
not be significant.
